\documentclass[禁則,森,便箋]{genkou}

\usepackage{pxbase}             % \UIを使用する,可以使用Unicode万国码插入生僻汉字
\usepackage[uplatex]{otf}

%%%% PDF 作者信息及超鏈接設定
\usepackage[dvipdfmx, 
    pdfdirection=R2L, % 開く方向%從右往左翻頁
    colorlinks=true,   %設置超鏈接的顔色
    linkcolor=blue,
    filecolor=blue,
    urlcolor=blue,
    citecolor=cyan,
    bookmarks=true, % PDFにしおりをつける
    bookmarksnumbered=true, % しおりに節番号などをつける
  ]{hyperref}
\hypersetup{ %
    pdftitle= {脂硯齋重評石頭記-庚辰本 ( the Tale of the Stone (GengChen Edition) )} , % PDFのタイトル
    pdfauthor= {曹雪芹( Cao Xue Qin )} , % PDFの作成者
    pdfkeywords = {Chinese Classical Novel },  %関鍵詞
    pdfsubject = {Chinese Classical Literature of Qing Dynasty},    % 主題
    pdfcreator  = {up\LaTeX\ with package  hyperref },    %工具
    pdfproducer = {dvipdfmx(20180506)},   %製作軟件
}

% PDFにしたときのしおりの文字化けを防ぐ  %使書簽支持CJK字體 内碼
\usepackage{pxjahyper}


\textwidth=60 zw



\begin{document}

\par \noindent  { \gtfamily \bfseries 月下獨酌 } \hfill {\gtfamily 唐\quad 李白} \hspace{50 mm}

\一字下げ 花間一壺酒,獨酌無相親。\\
舉杯邀明月,對影成三人。\\
月既不解飲,影徒隨我身。\\
暫伴月將影,行樂須及春。\\
我歌月徘徊,我舞影零亂。\\
醒時同交歡,醉後各分散。\\
永結無情遊,相期邈雲漢。

\clearpage

\par \noindent  { \gtfamily \bfseries 春望 } \hfill {\gtfamily 唐\quad 杜甫}  \hspace{50 mm}

國破山河在,城春草木深。\\
感時花濺淚,恨別鳥驚心。\\
烽火連三月,家書抵萬金。\\
白頭搔更短,渾欲不勝簪。\\


\end{document}
