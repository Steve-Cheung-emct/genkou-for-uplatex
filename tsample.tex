% 縦書き原稿用紙の見本です。
% 用紙をA4,横にしてプレビューしてください。
% これらの短歌は、見本用に平成太郎さん(E-Mail:NAH01433@nifty.ne.jp)に
% ご提供いただきました。
%
\documentclass[森,禁則,名前]{genkou}

%\textwidth=40 zw

\usepackage{pxbase}                           % \UIを使用する,可以使用Unicode万国码插入生僻汉字
\usepackage[uplatex]{otf}              %多書體支持

%%%% PDF 作者信息及超鏈接設定
\usepackage[dvipdfmx, 
    pdfdirection=R2L, % 開く方向%從右往左翻頁
    colorlinks=true,   %設置超鏈接的顔色
    linkcolor=blue,
    filecolor=blue,
    urlcolor=blue,
    citecolor=cyan,
    bookmarks=true, % PDFにしおりをつける
    bookmarksnumbered=true, % しおりに節番号などをつける
  ]{hyperref}
\hypersetup{ %
    pdftitle= {脂硯齋重評石頭記-庚辰本 ( the Tale of the Stone (GengChen Edition) )} , % PDFのタイトル
    pdfauthor= {曹雪芹( Cao Xue Qin )} , % PDFの作成者
    pdfkeywords = {Chinese Classical Novel },  %関鍵詞
    pdfsubject = {Chinese Classical Literature of Qing Dynasty},    % 主題
    pdfcreator  = {up\LaTeX\ with package  hyperref },    %工具
    pdfproducer = {dvipdfmx(20180506)},   %製作軟件
}

% PDFにしたときのしおりの文字化けを防ぐ  %使書簽支持CJK字體 内碼
\usepackage{pxjahyper}


\begin{document}

\gtfamily 

梢のみ秋に燃え立つ\ruby{洗玉燗}{せん,ぎょく,かん}忍ぶ恋
\ruby{佇}{た}つ\ruby{通天橋}{つう,てん,きょう}に\空行

時雨あと祇園\CID{119}丸山\ruby{散紅葉}{ちり,も,みじ}\ruby*{汝}{な}が手は握る
恋の\ruby{紅葉}{こう,やふ}\空行

山間の雀の鳴かぬ廃村に\ruby{祠}{ほこら}かたむき椎の実拾ふ\空行

母なるや我が\ruby{愛娘}{むす,め}見しきみが目は深きほほえみ吾にも向けん\空行

きみに似し洛北に咲くかきつばた指に触れたや花をもきみも\空行

春惜しむ旅の終わりや洛北の君と逢ふ瀬は苔光る寺\空行

立ちばやにきみが祈るる水芭蕉あくる夏もと契りをきして


\clearpage

{ \par \noindent \gtfamily \bfseries
第一回 甄士隱夢幻識通靈 賈雨村風塵懷閨秀
}

\ukai

  此開卷第一回也。作者自云:因曾歷過一番夢幻之後
,故將眞事隱去,而借通靈之說,撰此《石頭記》一書也。
故曰甄士隱云云。但書中所記何事何人?自又云:「今風
塵碌碌,一事無成,忽念及當日所有之女子,一一細考較
去,覺其行止見識,皆出於我之上。何我堂堂鬚眉,誠不
若彼裙釵哉?實愧則有餘,悔又無益之大無可如何之日也!
當此,則自欲將已往所賴 天恩祖德,錦衣紈絝之時,飫
甘饜肥之日,背父兄教育之恩,負師友規談之德,以至今
日一技無成,半生潦倒之罪,編述一集,以告天下人:我
之罪固不免,然閨閣中本自歷歷有人,萬不可因我之不肖,
自護己短,一併使其泯滅也。
雖今日之茆椽蓬牖,瓦竈繩床,其晨夕風露,階柳庭花,
亦未有防我之襟懷筆墨。雖我未學,下筆無文,又何妨用
假語村言,敷演出一段故事來,亦可使閨閣昭傳,復可以
悅世人之目,破人愁悶,不亦宜乎?」



\end{document}
